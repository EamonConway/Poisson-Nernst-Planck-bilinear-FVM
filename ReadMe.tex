% !TEX TS-program = pdflatex
% !TEX encoding = UTF-8 Unicode

% This is a simple template for a LaTeX document using the "article" class.
% See "book", "report", "letter" for other types of document.

\documentclass[11pt]{article} % use larger type; default would be 10pt

\usepackage[utf8]{inputenc} % set input encoding (not needed with XeLaTeX)

%%% Examples of Article customizations
% These packages are optional, depending whether you want the features they provide.
% See the LaTeX Companion or other references for full information.

%%% PAGE DIMENSIONS
\usepackage{geometry} % to change the page dimensions
\geometry{a4paper} % or letterpaper (US) or a5paper or....
% \geometry{margin=2in} % for example, change the margins to 2 inches all round
% \geometry{landscape} % set up the page for landscape
%   read geometry.pdf for detailed page layout information

\usepackage{graphicx} % support the \includegraphics command and options

% \usepackage[parfill]{parskip} % Activate to begin paragraphs with an empty line rather than an indent

%%% PACKAGES
\usepackage{booktabs} % for much better looking tables
\usepackage{array} % for better arrays (eg matrices) in maths
\usepackage{paralist} % very flexible & customisable lists (eg. enumerate/itemize, etc.)
\usepackage{verbatim} % adds environment for commenting out blocks of text & for better verbatim
\usepackage{subfig} % make it possible to include more than one captioned figure/table in a single float
\usepackage{tabu}
\usepackage{tabu}
% These packages are all incorporated in the memoir class to one degree or another...

%%% HEADERS & FOOTERS
\usepackage{fancyhdr} % This should be set AFTER setting up the page geometry
\pagestyle{fancy} % options: empty , plain , fancy
\renewcommand{\headrulewidth}{0pt} % customise the layout...
\lhead{}\chead{}\rhead{}
\lfoot{}\cfoot{\thepage}\rfoot{}

%%% SECTION TITLE APPEARANCE
\usepackage{sectsty}
\allsectionsfont{\sffamily\mdseries\upshape} % (See the fntguide.pdf for font help)
% (This matches ConTeXt defaults)

%%% ToC (table of contents) APPEARANCE
\usepackage[nottoc,notlof,notlot]{tocbibind} % Put the bibliography in the ToC
\usepackage[titles,subfigure]{tocloft} % Alter the style of the Table of Contents
\renewcommand{\cftsecfont}{\rmfamily\mdseries\upshape}
\renewcommand{\cftsecpagefont}{\rmfamily\mdseries\upshape} % No bold!

%%% END Article customizations

%%% The "real" document content comes below...

\title{Brief Article}
\author{The Author}
%\date{} % Activate to display a given date or no date (if empty),
         % otherwise the current date is printed 

\begin{document}
\abstract{This docum}
\section{Example}

\section{Function Overview}
%%
%bilinear\_system\_residual
%Call rr = bilinear\_system\_residual(inputs)
%Description
%Inputs
%Outputs
%
%Notes:
%{\fontfamily{pcr}\selectfont
%\noindent{\tabulinesep=2.5mm
% \begin{tabu}{ p{3.8cm} | p{8.2cm} | p{2cm}  }
%\hline Variable Name &Brief Description & Size\\ \hline 
%Parameters & Parameter Structure &  \\
%t & Current time & \\
%y & Current solution points & \\
%yp &Current temporal derivative &\\
%nodes & Node number and location& \\
%element & Element number and connectivity& \\
%Control\_Volume & Control Volume sizes& \\
%BoundaryElements & & \\
%DL & &\\
%\end{tabu}}
%\vspace{1cm}}

\noindent{\fontfamily{pcr}\selectfont
{\tabulinesep=2.5mm
 \begin{tabu}{ p{3cm} | p{11cm}  }
\hline Name & bilinear\_system\_residual\\ \hline 
Call & rr = bilinear\_system\_residual(t, y, yp, Control\_Volume, Parameters, element, nodes, BoundaryElements) \\
Description & This function evaluates the residual of the discretised Poisson-Nernst-Planck equations.\\
Inputs &\begin{tabu}{p{2cm} p{9cm}} t & Current time point. \\ 
y & Vector of size (3*length(nodes)) consisting of the current solution values for each discretised.  \\
yp & Vector of size (3*length(nodes)) consisting of the current time derivatives for each discretised equation.  \\
 Parameters & Structure defining all parameters required in simulation. \\ element & Mesh connectivity matrix.\\ nodes & Node numbers and location of nodes\\ 
 DL & Matrix defining the length of control volume faces in each element. \end{tabu} \\
Outputs & \begin{tabu}{p{2cm} p{9cm}} rr & A vector of size (3*length(nodes)) that returns the value of the residual for each equation. \end{tabu} \\   \hline
Notes &  \\ \hline
\end{tabu}}
\vspace{1cm}}

{\fontfamily{pcr}\selectfont
\noindent{\tabulinesep=2.5mm
 \begin{tabu}{ p{3cm} | p{11cm}  }
\hline Name & bilinear\_flux\_mex\\ \hline 
Call & Flux = bilinear\_flux\_mex (t, y, Parameters, element, nodes, DL) \\
Description & BILINEAR\_FLUX\_MEX evaluates the integral of the flux term in the discretised Poisson-Nernst-Planck equations.\\
Inputs & 

\begin{tabu}{p{2cm} p{9cm}} t & Current time point. \\ 
y & A vector of size (3*length(nodes)) consisting of the current solution values at each node point.  \\
 Parameters & Structure defining all parameters required in simulation. \\ element & Mesh connectivity matrix.\\ nodes & Node numbers and location of nodes\\ 
 DL & Matrix defining the length of control volume faces in each element. \end{tabu} \\

Outputs &  \begin{tabu}{p{2cm} p{9cm}} Flux & Vector of size (3*length(nodes)) that returns surface integral of the flux around each node in the mesh\end{tabu}\\
& \\ \hline
Notes & 

%\begin{tabu}{p{11cm}} The vector y is of length 3*length(nodes) and structured such that y(1:3:end) = C$_1$, y(2:3:end) = C$_2$ and y(3:3:end) = Phi, where C$_1$, C$_2$ are the concentration of species one and two respectively, and Phi is the potential. \\
%\\
%The matrix element is of size (number of elements)x5 and is structured such that each row consists of the element number, followed by the node number of each vertex.The defintion of the first vertex of the element is arbitrary, but subsequent vertices must be defined in a counter clockwise direction.  \\ \\  \end{tabu} 

      \\ \hline
\end{tabu}}
\vspace{1cm}}


{\fontfamily{pcr}\selectfont
\noindent{\tabulinesep=2.5mm
 \begin{tabu}{ p{3cm} | p{11cm}  }
\hline Name & Control\_Volume\_function \\ \hline 
Call & Control\_Volume = Control\_Volume\_function(element, nodes)\\
Description & CONTROL\_VOLUME\_FUNCTION calculates the volume of each control volume in the mesh defined by element and nodes.\\
Inputs & \begin{tabu}{p{2cm} p{9cm}} 
element &  Mesh connectivity matrix. \\
nodes & Node numbers and location of nodes 
\end{tabu} \\
Outputs &  \begin{tabu}{p{2cm} p{9cm}}  & \end{tabu}\\
& \\ \hline
Notes & \begin{tabu}{p{11cm}} \end{tabu}    \\ \hline
\end{tabu}}
\vspace{1cm}}


{\fontfamily{pcr}\selectfont
\noindent{\tabulinesep=2.5mm
 \begin{tabu}{ p{3cm} | p{11cm}  }
\hline Name & jacobian\_calc(nodes, element)\\ \hline 
Call & [dfdy, dfdyp] =  jacobian\_calc(nodes, element) \\
Description & JACOBIAN\_CALC evaluates the connectivity between each element in the jacobian. This is used to efficiently calculate the Jacobian in the ode solver. \\
Inputs & \begin{tabu}{p{2cm} p{9cm}}  nodes & Node numbers and location of nodes\\ 
element & Mesh connectivity matrix.
 \end{tabu} \\
Outputs &  \begin{tabu}{p{2cm} p{9cm}}  & \end{tabu}\\
& \\ \hline
Notes & \begin{tabu}{p{11cm}} \end{tabu}    \\ \hline
\end{tabu}}
\vspace{1cm}}

{\fontfamily{pcr}\selectfont
\noindent{\tabulinesep=2.5mm
 \begin{tabu}{ p{3cm} | p{11cm}  }
\hline Name & length\_element\\ \hline 
Call &length =  length\_element(x, y)\\
Description & LENGTH\_ELEMENT calculates the length of each face inside the quadrilateral with vertices defined by x and y.  \\
Inputs & \begin{tabu}{p{2cm} p{9cm}} x & Horizontal coordinate \\ y & Vertical coordinate  \end{tabu} \\
Outputs &  \begin{tabu}{p{2cm} p{9cm}}  & \end{tabu}\\
& \\ \hline
Notes & \begin{tabu}{p{11cm}} \end{tabu}    \\ \hline
\end{tabu}}
\vspace{1cm}}



{\fontfamily{pcr}\selectfont
\noindent{\tabulinesep=2.5mm
 \begin{tabu}{ p{3cm} | p{11cm}  }
\hline Name & unsteady\_bilinear\\ \hline 
Call & Unsteady = unsteady\_bilinear(t, y, yp, Parameters, element, nodes) \\
Description & UNSTEADY\_BILINEAR evaluates the unsteady term in the finite volume discretisation of the Poisson-Nernst-Planck equations.\\
Inputs & \begin{tabu}{p{2cm} p{9cm}} t & Current time point. \\ 
y & A vector of size (3*length(nodes)) consisting of the current solution values at each node point.  \\
yp & Vector of size (3*length(nodes)) consisting of the current time derivatives for each discretised equation.  \\
 Parameters & Structure defining all parameters required in simulation. \\ element & Mesh connectivity matrix.\\ nodes & Node numbers and location of nodes \end{tabu} \\
Outputs &  \begin{tabu}{p{2cm} p{9cm}}  & \end{tabu}\\
& \\ \hline
Notes & \begin{tabu}{p{11cm}} \end{tabu}    \\ \hline
\end{tabu}}
\vspace{1cm}}


{\fontfamily{pcr}\selectfont
\noindent{\tabulinesep=2.5mm
 \begin{tabu}{ p{3cm} | p{11cm}  }
\hline Name &source\_bilinear\\ \hline 
Call &  Source = source\_bilinear(t, y, yp, Parameters, element, nodes)\\
Description & SOURCE\_BILINEAR evaluates the source term in the finite volume discretisation of the Poisson-Nernst-Planck equations\\
Inputs &\begin{tabu}{p{2cm} p{9cm}} t & Current time point. \\ 
y & A vector of size (3*length(nodes)) consisting of the current solution values at each node point.  \\
yp & Vector of size (3*length(nodes)) consisting of the current time derivatives for each discretised equation.  \\
 Parameters & Structure defining all parameters required in simulation. \\ element & Mesh connectivity matrix.\\ nodes & Node numbers and location of nodes \end{tabu} \\
Outputs &  \begin{tabu}{p{2cm} p{9cm}}  & \end{tabu}\\
& \\ \hline
Notes & \begin{tabu}{p{11cm}} \end{tabu}    \\ \hline
\end{tabu}}
\vspace{1cm}}


\end{document}






















